\begin{fullwidth}
    Integriamo la sezione d'urto di scattering sullo schermo $ S_O$ su cui si osserva la diffrazione in approssimazione di
    far field(schermo lontano).
    Supporremo inoltre lo schermo molto esteso rispetto all'estensione della figura di diffrazione.
    \[
        \sigma_{scatt} = \iint_{\Omega} |f(\bm{q})|^2 \, d \Omega = \iint_{S_O} \frac{|f(\bm{q})|^2}{r^2} \, da
    \]
    Ora, sostituendo l'espressione della scattering amplitude (\ref{eq:scattering-amplitude}) ed utilizzando l'identità
    $ |f|^2 = \bar{f}f$ abbiamo
    \begin{align}
        \sigma_{scatt} &= \iint_{S_O} \left[
            \frac{1}{{r}^{2}} \left(
            \frac{ik}{2 \pi} \iint_{S_B} \Gamma(\bm{r}')
            e^{-i \bm{q} \cdot \bm{r}'} \, da'
            \right)
            \left( - \frac{ik}{2 \pi} \iint_{S_B} \bar{\Gamma}(\bm{r}') e^{i \bm{q} \cdot \bm{r}''} \, da'' \right)
            \right] \,  da  \nonumber \\
        & =  \frac{{k}^{2}}{4 \pi^2} \iint_{S_O} \left( \frac{1}{{r}^{2}} \iint_{S_B}  \iint_{S_B}
        \Gamma(\bm{r}') \bar{\Gamma}(\bm{r}'') e^{i \bm{q} \cdot (\bm{r}''-\bm{r}')} \, da'' da' \right) \, da \nonumber\\
        & = \frac{{k}^{2}}{4 \pi^2}  \iint_{S_B}  \iint_{S_B}\left(\iint_{S_O} \frac{1}{{r}^{2}}
        \Gamma(\bm{r}') \bar{\Gamma}(\bm{r}'') e^{i \bm{q} \cdot (\bm{r}''-\bm{r}')} \, da \right) \, da'' da'
        \label{eq:sigma-scattering-1}
    \end{align}
    Vogliamo ora sviluppare la fase dell'esponenziale.
    Assumendo un sistema di coordinate cartesiano con l’asse z orientato normalmente agli schermi, la coordinata z
    del secondo schermo assume valori grandi, si hanno allora le relazioni seguenti
    \[
    r = \sqrt{ x^{2}+y^{2}+z^{2} } = z \sqrt{ 1+ \frac{x^{2}+y^{2}}{z^{2}} } \simeq z \left( 1 + \frac{x^{2}+y^{2}}{2z^{2}} \right)
    = z + \frac{x^{2}+y^{2}}{2z} \simeq z
    \]
    Per quanto riguarda il vettore d'onda trasferito, definendo $\hat{ \bm{n}} = \frac{\bm{r}}{r}$ nell'approssimazione
    di $z \gg x,y$ si ha
    \begin{gather*}
        \bm{q} = k (\hat{\bm{n}} - \hat{\bm{n}}') = k \left( \frac{\bm{r}}{r}  - \hat{\bm{k}}\right) \simeq k\left( \frac{x}{z}\hat{\bm{i}} + \frac{y}{z} \hat{\bm{j}} + \hat{\bm{k}} -\hat{\bm{k}}\right) = k\left( \frac{x}{z} \hat{\bm{i}} +\frac{y}{z} \hat{\bm{j}} \right)\\
        \bm{q} \cdot (\bm{r}'' - \bm{r}'') \simeq \left( \frac{x}{z} \hat{\bm{i}} + \frac{y}{z} \hat{\bm{j}} \right) \cdot [(x''-x')\hat{\bm{i}} + (y''-y')\hat{\bm{j}}+(z''-z')\hat{\bm{k}}]
    = \frac{kx}{z}(x''-x') +\frac{ky}{z}(y''-y')
    \end{gather*}
    che sostituite nella (\ref{eq:sigma-scattering-1}) forniscono
    \[
    \sigma_{scatt} = \frac{k^{2}}{4 \pi^{2}} \iint_{S_{B}}\iint_{S_{B}} \left( \iint_{S_{O}} \frac{1}{z^{2}} \Gamma(\bm{r}')\bar{\Gamma}(\bm{r}'') e^{ i \frac{kx}{z}(x''-x') + i \frac{ky}{z} (y'' - y')} \, dxdy\right) da''da'
    \]
    Nelle nostre ipotesi $S_{o} \gg S_{B}$, per cui sviluppiamo la precedente come
    \[
     = \frac{k^{2}}{4 \pi^{2}} \iint_{S_{B}}\iint_{S_{B}} da'' da'\frac{1}{z^{2}} \Gamma(\bm{r}') \bar{\Gamma}(\bm{r}'') \frac{4 \pi^{2}z^{2}}{k^{2}}\left( \frac{1}{2 \pi} \int_{-\infty}^{\infty} e^{ ik \frac{x}{z}(x'-x'') } \frac{x}{z} \, dk  \right)\left( \frac{1}{2 \pi} \int_{-\infty}^{\infty} e^{ ik \frac{y}{z}(y'-y'') } \frac{y}{z} \, dk  \right)
    \]
    Utilizzando ora l'espressione esponenziale della delta di Dirac:
    \[
    \int_{-\infty}^{\infty} e^{ ikx } \, dk = 2 \pi \, \delta(x)
    \]
    riscriviamo
    \[
    \sigma_{scatt} = \iint_{S_{B}}\iint_{S_{B}} \Gamma(\bm{r}') \bar{\Gamma}(\bm{r}'') \delta(x''-x')\delta(y''-y') \,dx''dy''dx'dy'
    \]
    Tenendo ora conto di
    \begin{itemize}
        \item $z'' = z'$ avendo scelto il riferimento con l'asse $z$ normale al piano di integrazione;
        \item le delta di dirac bloccano gli integrali ai valori $x''=x',y''=y'$;
        \item $\bm{r}''=\bm{r}'$(gli assi $z$ e $z'$ sono collineari). %%%TODO 12/12/22 niccolozanotti: spiegare perche. Spiega bene il fatto che |f|^2 = fbar * f di stessa variabile --> poi
    \end{itemize}
    in defininita abbiamo
    \[
    \sigma_{scatt} = \iint_{S_{B}}  \Gamma(\bm{r}') \bar{\Gamma}(\bm{r}'') \, dx'dy'= \iint_{S_{B}} |\Gamma(\bm{r}')|^{2} \, dx'dy'
    \]
    Abbiamo quindi ritrovato la (\ref{eq:scattering-cross-section-profile-func}):
    \[
    \sigma_{scatt} = \iint_{S_{B}}  |\Gamma(\bm{r}')|^{2} \, da'
    \]
\end{fullwidth}
